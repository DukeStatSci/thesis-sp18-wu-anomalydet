% This is the Duke University Statistical Science LaTeX thesis template.
% It has been adapted from the Reed College LaTeX thesis template. The
% adaptation was done by Mine Cetinkaya-Rundel (MCR). Some of the comments
% that are specific to Reed College have been removed.
%
% Most of the work on the original Reed College document class and template
% was done by Sam Noble (SN). Later comments etc. by Ben Salzberg (BTS).
% Additional restructuring and APA support by Jess Youngberg (JY).
%
% See https://www.reed.edu/cis/help/latex/ for help. There are a
% great bunch of help pages there, with notes on
% getting started, bibtex, etc. Go there and read it if you're not
% already familiar with LaTeX.
%
% Any line that starts with a percent symbol is a comment.
% They won't show up in the document, and are useful for notes
% to yourself and explaining commands.
% Commenting also removes a line from the document;
% very handy for troubleshooting problems. -BTS

%%
%% Preamble
%%
% \documentclass{<something>} must begin each LaTeX document
\documentclass[12pt,twoside]{dukestatscithesis}
% Packages are extensions to the basic LaTeX functions. Whatever you
% want to typeset, there is probably a package out there for it.
% Chemistry (chemtex), screenplays, you name it.
% Check out CTAN to see: http://www.ctan.org/
%%
\usepackage{graphicx,latexsym}
\usepackage{amsmath}
\usepackage{amssymb,amsthm}
\usepackage{longtable,booktabs,setspace}
\usepackage{chemarr} %% Useful for one reaction arrow, useless if you're not a chem major
\usepackage[hyphens]{url}
% Added by CII
\usepackage{hyperref}
\usepackage{lmodern}
\usepackage{float}
\floatplacement{figure}{H}
% End of CII addition
\usepackage{rotating}

% Next line commented out by CII
%%% \usepackage{natbib}
% Comment out the natbib line above and uncomment the following two lines to use the new
% biblatex-chicago style, for Chicago A. Also make some changes at the end where the
% bibliography is included.
%\usepackage{biblatex-chicago}
%\bibliography{thesis}


% Added by CII (Thanks, Hadley!)
% Use ref for internal links
\renewcommand{\hyperref}[2][???]{\autoref{#1}}
\def\chapterautorefname{Chapter}
\def\sectionautorefname{Section}
\def\subsectionautorefname{Subsection}
% End of CII addition

% Added by CII
\usepackage{caption}
\captionsetup{width=5in}
% End of CII addition

% \usepackage{times} % other fonts are available like times, bookman, charter, palatino


% To pass between YAML and LaTeX the dollar signs are added by CII
\title{}
\author{}
% The month and year that you submit your FINAL draft TO THE LIBRARY (May or December)
\date{}
\advisor{}
\institution{}
\degree{}
\committeememberone{}
\committeemembertwo{}
\dus{}
%If you have two advisors for some reason, you can use the following
% Uncommented out by CII
% End of CII addition

%%% Remember to use the correct department!
\department{}

% Added by CII
%%% Copied from knitr
%% maxwidth is the original width if it's less than linewidth
%% otherwise use linewidth (to make sure the graphics do not exceed the margin)
\makeatletter
\def\maxwidth{ %
  \ifdim\Gin@nat@width>\linewidth
    \linewidth
  \else
    \Gin@nat@width
  \fi
}
\makeatother

\renewcommand{\contentsname}{Table of Contents}
% End of CII addition

\setlength{\parskip}{0pt}

% Added by CII

\providecommand{\tightlist}{%
  \setlength{\itemsep}{0pt}\setlength{\parskip}{0pt}}

\Acknowledgements{

}

\Dedication{

}

\Preface{

}

\Abstract{

}

% End of CII addition
%%
%% End Preamble
%%
%
\begin{document}

% Everything below added by CII

\frontmatter % this stuff will be roman-numbered
\pagestyle{empty} % this removes page numbers from the frontmatter



  \hypersetup{linkcolor=black}
  \setcounter{tocdepth}{2}
  \tableofcontents





\mainmatter % here the regular arabic numbering starts
\pagestyle{fancyplain} % turns page numbering back on

\chapter*{Preliminary Content}\label{preliminary-content}
\addcontentsline{toc}{chapter}{Preliminary Content}

\section*{Acknowledgements}\label{acknowledgements}
\addcontentsline{toc}{section}{Acknowledgements}

I want to thank my Advisor, Professor Peter Hoff, and the Director of
Undergraduate Studies, Professor Mine Cetinkaya-Rundel, for their
guidance in this project. I also want to thank my parents for their
continued unwavering support in all my endeavors.

\section*{Abstract}\label{abstract}
\addcontentsline{toc}{section}{Abstract}

The goal of this project is to identify novel methods for detecting
anomalies in network IP data. The space is represented as a
3-dimensional tensor of the continuous features (source bytes,
destination bytes, source packets, destination packets) divided by their
respective source port and destination port combinations. This project
implements and assesses the validity of principal component analysis and
matrix completion via singular value decomposition (more methods
pending) in determining anomalous entries in the tensor.

\chapter{Introduction}\label{introduction}

\section{Anomaly Detection}\label{anomaly-detection}

\subsection{Supervised versus
Unsupervised}\label{supervised-versus-unsupervised}

\section{Network Attacks}\label{network-attacks}

Network security is becoming increasingly relevant as the flow of data,
bandwith of transactions, and user dependency on hosted networks
increase. As entire networks grow in nodes and complexity, attackers
gain easier entry points of access to the network. The most benign of
attackers attempt to shutdown networks (e.g.~causing a website to
shutdown with repeated pings to its server), while more malicious
attempts involve hijacking the server to publish the attacker's own
content or stealing unsecured data from the server, thus compromising
the privacy of the network's users.

Attackers follow a specific three step strategy when gathering
intelligence on a network, the most important component of which is
scanning. Network scanning is a procedure for identifying active hosts
on a network, the attacker uses it to find information about the
specific IP addresses that can be accessed over the Internet, their
target's operating systems, system architecture, and the services
running on each node/computer in the network. Scanning procedures, such
as ping sweeps and port scans, return information about which IP
addresses map to live hosts that are active on the Internet and what
services they offer. Another scanning method, inverse mapping, returns
information about what IP addresses do not map to live hosts; this
enables an attacker to make assumptions about viable addresses.

All three of these scanning methods leave digital signatures in the
networks they evaluate because they apply specific pings that are then
stored in the network logs. Most scanners use a specific combination of
bytes, packets, flags (in TCP protocol), and ports in a sequence of
pings to a network. Identifying a scanner's often many IP addresses from
the set of pings available in the network's logs is thus an unsupervised
anomaly detection problem. \(\newline\) This particular dataset is from
Duke University's Office of Information Technology, and it covers all
transactions in their network traffic during a five minute period in
February 2017.

\chapter{Literature Review}\label{literature-review}

\section{Name of Current OIT Method}\label{name-of-current-oit-method}

\section{Kernel Principal Component
Analysis}\label{kernel-principal-component-analysis}

\section{Matrix Completion via Singular Value
Decomposition}\label{matrix-completion-via-singular-value-decomposition}

Ask Mine what goes into a literature review

--\textgreater{}

\chapter{Network Attacks Dataset}\label{network-attacks-dataset}

\section{Features}\label{features}

\subsection{Argus}\label{argus}

\subsection{Categorical Features}\label{categorical-features}

\subsection{Continuous Features}\label{continuous-features}

\section{Exploratory Data Analysis}\label{exploratory-data-analysis}

\section{Relationships}\label{relationships}

\section{Correlation}\label{correlation}

\section{Other Stuff}\label{other-stuff}

Does EDA even go into a thesis or is it appendix?

--\textgreater{}

\chapter{Matrix Techniques for Anomaly
Detection}\label{matrix-techniques-for-anomaly-detection}

\section{Ports Combination
Matrix/Tensor}\label{ports-combination-matrixtensor}

\section{Principal Component
Analysis}\label{principal-component-analysis}

\section{Matrix Completion via Singular Value
Decomposition}\label{matrix-completion-via-singular-value-decomposition-1}

--\textgreater{}

--\textgreater{}

--\textgreater{}

--\textgreater{}

\chapter{Statistical Model}\label{statistical-model}

\section{Uneven Variances}\label{uneven-variances}

Discuss AMMI model

\section{Determined Model}\label{determined-model}

\section{}\label{section}

--\textgreater{}

--\textgreater{}

--\textgreater{}

--\textgreater{}

\chapter*{Conclusion}\label{conclusion}
\addcontentsline{toc}{chapter}{Conclusion}

If we don't want Conclusion to have a chapter number next to it, we can
add the \texttt{\{-\}} attribute.

\textbf{More info}

And here's some other random info: the first paragraph after a chapter
title or section head \emph{shouldn't be} indented, because indents are
to tell the reader that you're starting a new paragraph. Since that's
obvious after a chapter or section title, proper typesetting doesn't add
an indent there.

\appendix

\chapter{The First Appendix}\label{the-first-appendix}

This first appendix includes all of the R chunks of code that were
hidden throughout the document (using the \texttt{include\ =\ FALSE}
chunk tag) to help with readibility and/or setup.

\textbf{In the main Rmd file}
\begin{Shaded}
\begin{Highlighting}[]
\CommentTok{# This chunk ensures that the thesisdowndss package is}
\CommentTok{# installed and loaded. This thesisdowndss package includes}
\CommentTok{# the template files for the thesis.}
\NormalTok{if(!}\KeywordTok{require}\NormalTok{(devtools))}
  \KeywordTok{install.packages}\NormalTok{(}\StringTok{"devtools"}\NormalTok{, }\DataTypeTok{repos =} \StringTok{"http://cran.rstudio.com"}\NormalTok{)}
\NormalTok{if(!}\KeywordTok{require}\NormalTok{(thesisdowndss))}
  \NormalTok{devtools::}\KeywordTok{install_github}\NormalTok{(}\StringTok{"mine-cetinkaya-rundel/thesisdowndss"}\NormalTok{)}
\KeywordTok{library}\NormalTok{(thesisdowndss)}
\end{Highlighting}
\end{Shaded}
\textbf{In Chapter \ref{ref-labels}:}

\chapter{The Second Appendix, for
Fun}\label{the-second-appendix-for-fun}

\backmatter

\chapter*{References}\label{references}
\addcontentsline{toc}{chapter}{References}

\markboth{References}{References}

\noindent

\setlength{\parindent}{-0.20in} \setlength{\leftskip}{0.20in}
\setlength{\parskip}{8pt}

\hypertarget{refs}{}
\hypertarget{ref-angel2000}{}
Angel, Edward. 2000. \emph{Interactive Computer Graphics : A Top-down
Approach with Opengl}. Boston, MA: Addison Wesley Longman.

\hypertarget{ref-angel2001}{}
---------. 2001a. \emph{Batch-File Computer Graphics : A Bottom-up
Approach with Quicktime}. Boston, MA: Wesley Addison Longman.

\hypertarget{ref-angel2002a}{}
---------. 2001b. \emph{Test Second Book by Angel}. Boston, MA: Wesley
Addison Longman.


% Index?

\end{document}
